\addcontentsline{toc}{section}{ZUSAMMENFASSUNG}
\section*{ZUSAMMENFASSUNG}
Darlegung des Themas, der Fragestellung, der Problemformulierung sowie der Ergebnisse.
Kurze, prägnante Information über den Inhalt der Arbeit; keine Wertungen oder Meinungen (allerdings in ganzen Sätzen!).

\subsection*{A Aufgabenstellung}
\begin{itemize}
	\item Von welchem Wissens- oder Entwicklungsstand im Umfeld der Aufgabenstellung wird ausgegangen bzw. welche Ergebnisse und Erkenntnisse gibt es bereits zum Thema?
	\item Welches Ziel soll erreicht werden?
	\item Warum und für wen ist das definierte Ziel von Interesse?
\end{itemize}

\subsection*{B Umsetzung}
\begin{itemize}
	\item Auf welche fachtheoretischen/-praktischen Grundlagen wurde zurückgegriffen?
	\item Welche Lösungsansätze/Methoden wurden gewählt?
	\item Warum gerade diese und keine anderen?
	\item Welche Alternativen gäbe es noch?
	\item Es könnte ev. auf den Bearbeitungsprozess im Team eingegangen werden (Reflexion).
\end{itemize}

\subsection*{C Ergebnisse}
\begin{itemize}
	\item Worin besteht der Beitrag zur Lösung der Aufgabenstellung? (Website-Erstellung, Marketingkonzept, …)
	\item Welches Produkt soll erstellt werden?
	\item Es könnte ev. darauf eingegangen werden, ob die Diplomarbeit bei Wettbewerben eingereicht wurde oder ob es Prämierungen gab.
\end{itemize}


Das sind Fragestellungen, die in der ZUSAMMENFASSUNG und auch in der Übersetzung in englischer Sprache im ABSTRACT vorkommen können (als Vorschlag – je nach Projekt ergeben sich auch andere Fragen).
Die Struktur mit A Aufgabenstellung, B Umsetzung und C Ergebnisse muss aber jedenfalls eingehalten werden!
