\addcontentsline{toc}{section}{DIPLOMARBEIT DOKUMENTATION}
\section *{DIPLOMARBEIT DOKUMENTATION}
	\SecAuth{\emplLastA} % festlegen der Verfasser dieses Kapitels für die Fußzeile

\begin{tabular}{@{}p{5cm}p{8cm}}
Name der Verfasser~\textbar~innen & \emplA, \emplB, \emplC, \emplD \\

Jahrgang ~\textbar~ Schuljahr & 5Klasse ~\textbar~ 20SJ~\textbar~20SJ \\

THEMA der Diplomarbeit & Spiel- und Kontrollerentwicklung \\

Aufgabenstellung & Es ist ein Videospiel welches mit einem Kontroller einfach zu bedienen ist. Dabei soll auch eine kleine Geschichte mit dem Spiel erzählt werden. Die Diplomarbeit soll auch ein How-To für Erstellung eines Unity-Spiels gesteuert durch Megacard und ESP Kontrollers sein.  \\
\end{tabular}

\pagebreak

\subsection *{Individuelle Themenstellung im Rahmen des Gesamtprojektes}
\begin{tabular}{@{}p{5cm}p{8cm}}

	\emplA & 	Herr Burcak macht das Projektmanagement des Projektes, die Auslesung der Kontroller Signale und das Audio, Story, Benutzeroberfläche, Webserver und den Multiplayer-Teil des Spiels.  \\
		
	\emplB & 	Die grundlegenden Bewegungen vom Charakter und Gegner, sowie die Physik des Spiels werden von Semih Sönmez programmiert. Auch die Umgebung mit Missionen und Levels wird vom Herrn Sönmez entwickelt.  \\
		
	\emplC & 	 
	
	Manuel Rath ist für die Steuerung des Spiels zuständig. Der Kontroller wird aus der Megacard von der Schule erstellt. Die Kommunikation zwischen Kontroller und Spiel erfolgt über USB oder Bluetooth.  \\

	\emplD & 	Herr Herceg entwickelt den 2. Kontroller für die Steuerung. Die Daten werden mittels ESP Moduls per WLAN oder Bluetooth an den PC gesendet. Zudem werden für beide Kontroller ein Gehäuse und Stromversorgung angelegt.  \\

Realisierung & Bei Realisierung angeben, sonst freilassen.  \\

Ergebnisse & Das Spiel ist in Unity-3D zu realisieren. Die Highscores werden auf einem zentralen Server gespeichert. Die Eingaben vom Kontroller sollen von der Megacard/ESP ausgelesen und mittels USB/WLAN/Bluetooth an das Spiel weitergeitet werden und die Spielfigur soll sich entsprechend bewegen. Der aus einem eigenen Akku und Gehäuse bestehender Kontroller soll auch als Feedback beim Spielen vibrieren.   \\

Einsichtnahmen**) & Archiv der HTL Rankweil, \newline www.diplomarbeiten.berufsbildendeschulen.at \\
\end{tabular}
